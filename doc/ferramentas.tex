\chapter{Tecnologias e Ferramentas Utilizadas}
\label{ch::tecno}

\section{Introdução}
\label{sec::tecno:intro}

Este capítulo enumera as tecnologias e ferramentas utilizadas para o desenvolvimento da parte de inteligência de um \emph{robot} num mundo virtual, projeto descrito neste documento.


\section{Ferramentas Utilizadas}
\label{sec::tecno:ferr}

São descritas em seguida, primariamente, as ferramentas e tecnologias utilizadas, acrescentando também a razão pelas quais as mesmas foram escolhidas.

\begin{enumerate}
    \item Biblioteca \emph{NetworkX} \cite{NetworkX} --- biblioteca do \emph{Python} \cite{Python} utilizada para criar e manipular as estruturas de dados utilizadas (grafos);
    
    \item \emph{Git} --- sistema de controlo de versões no qual foi gerido o repositório do código-fonte do projeto;
    
    \item \emph{Python} \cite{Python} --- linguagem de programação utilizada para o desenvolvimento do projeto.
    
    % \item \emph{Visual Studio Code} \cite{Code} --- editor de texto utilizado para a escrita do código-fonte do projeto;
\end{enumerate}


\section{Conclusões}
\label{sec::tecno:conc}

Neste capítulo foram descritas as tecnologias e ferramentas utilizadas na realização e desenvolvimento do projeto referido neste documento. As mesmas são referidas nos próximos capítulos para efeitos de explicação de como as mesmas foram aplicadas.