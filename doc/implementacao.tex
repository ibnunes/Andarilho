\chapter{Desenvolvimento e Implementação}
\label{ch::implement}

\section{Introdução}
\label{sec::implement:intro}

Este capítulo explora o percurso realizado aquando do desenvolvimento do projeto descrito neste documento, em particular as escolhas e detalhes de implementação.


\section{Escolhas de Implementação}
\label{sec::implement:struct}

Antes da implementação dos algoritmos para satisfazer as respostas às perguntas colocadas, foi necessário definir a estrutura de dados utilizada para a manipulação do mundo virtual e dos objetos nele contidos.

Desta forma, além da escolha do uso de listas para guardar os objetos que o dispositivo robótico encontra no seu mundo virtual, foi utilizada numa primeira abordagem uma matriz $800\times600$ para efeitos de aplicação de um algoritmo de \textit{path-finding} (necessário para as respostas aos enunciados descritos nas subseccções \ref{chap3:subsec:perg3}, \ref{chap3:subsec:perg4} e \ref{chap3:subsec:perg5}). Esta abordagem revelou alguns problemas de eficiência, conforme é refletido na secção \ref{sec::reflexao:problemas}). A solução passou então pela substituição da matriz por uma estrutura de dados que permite representar de forma bastante eficiente os dados recolhidos sobre o mundo virtual: grafos. Para este fim, foi utilizada a biblioteca \emph{NetworkX} \cite{NetworkX} para a criação e gestão dos mesmos. Dois grafos foram utilizados neste âmbito:

\begin{enumerate}
	\item Grafo \emph{floor} (Figura \ref{fig::grafo_floor}) --- armazena informações sobre as salas visitadas, a sua ligação e os objetos nelas contidos;
	
	\item Grafo \emph{map} (Figura \ref{fig::grafo_map}) --- realiza o mapeamento do mundo, ao registar em detalhe todos os caminhos possíveis entre salas e as portas que as conectam, para efeitos de execução do algoritmo de \textit{path-finding} A*.
\end{enumerate}

\begin{figure}[!htb]
	% TODO: Pôr imagem do grafo floor
	\centering
	%\includegraphics[width=\textwidth]{}
	\caption{}{}
	\label{fig::grafo_floor}
\end{figure}

Por análise do grafo da Figura \ref{fig::grafo_floor}, podemos constatar que se trata de um grafo não dirigido, composto por nodos e arestas, sendo que os nodos guardam informações acerca das divisões e como seus atributos constam os objetos contidos nessas mesmas divisões. Os atributos estão definidos como um dicionário, onde a chave do mesmo é a categoria de objeto e o valor é a lista de nomes dos objetos dessa categoria encontrados pelo \emph{robot}.

\begin{figure}[!htb]
	% TODO: Pôr imagem do grafo map
	\centering
	%\includegraphics[width=\textwidth]{}
	\caption{}{}
	\label{fig::grafo_map}
\end{figure}

% TODO: Falar sobre o grafo map

% No decorrer de todo o projeto foi utilizado um modo de \emph{debugging} com \emph{logs} para efeitos de maior facilidade na correção de erros no código.

Em termos de organização do código-fonte do trabalho prático, optou-se por se utilizar classes, as quais se enumeram seguidamente:

\begin{enumerate}
	\item \emph{Log} --- útil para efeitos de \emph{debugging} de forma a identificar mais facilmente possíveis \emph{bugs};
	
	\item \emph{LinearFunction} --- permite criar instâncias de funções lineares necessárias para a resposta às perguntas 5 e 6 (subsecções \ref{ssec::implement:details:perg5} e \ref{ssec::implement:details:perg6}, respetivamente);
	
	\item \emph{Things} --- armazena listas de objetos e pessoas, tratando diretamente da resposta à pergunta 1 (subsecção \ref{ssec::implement:details:perg1}) e auxiliando as restantes classes a gerir os seus dados;
	
	\item \emph{Robot} --- realiza a gestão dos dados inerentes ao \textit{robot}, em particular a velocidade e a bateria, bem como a sua relação com o tempo;
	
	\item \emph{Hospital} --- principal classe do programa na qual a informação relativa ao piso do hospital é atualizada conforme as informações dadas pelo \emph{robot};
	
	\item \emph{Utils} --- coleta um conjunto de funções auxiliares, como por exemplo o cálculo de distâncias, a troca de variáveis e a descrição textual de caminhos.
\end{enumerate}


\section{Detalhes de Implementação}
\label{sec::implement:details}

Nesta secção são descritos os detalhes de implementação para cada pergunta proposta no enunciado do projeto prático, através da explicação dos respetivos algoritmos.


\subsection{Pergunta 1}
\label{ssec::implement:details:perg1}

Esta pergunta tem como seu enunciado ``Qual foi a penúltima pessoa que viste?''.


\subsection{Pergunta 2}
\label{ssec::implement:details:perg2}

Esta pergunta tem como seu enunciado ``Em que tipo de sala estás agora?''.


\subsection{Pergunta 3}
\label{ssec::implement:details:perg3}

Esta pergunta tem como seu enunciado ``Qual o caminho até à sala de enfermeiros mais próxima?''.


\subsection{Pergunta 4}
\label{ssec::implement:details:perg4}

Esta pergunta tem como seu enunciado ``Qual é a distância até ao médico mais próximo?''.


\subsection{Pergunta 5}
\label{ssec::implement:details:perg5}

Esta pergunta tem como seu enunciado ``Quanto tempo achas que demoras a ir de onde estás até às escadas''.


\subsection{Pergunta 6}
\label{ssec::implement:details:perg6}

Esta pergunta tem como seu enunciado ``Quanto tempo achas que falta até ficares sem bateria''.


\subsection{Pergunta 7}
\label{ssec::implement:details:perg7}

Esta pergunta tem como seu enunciado ``Qual a probabilidade de encontrar um livro numa divisão se já encontraste uma cadeira?''.


\subsection{Pergunta 8}
\label{ssec::implement:details:perg8}

Esta pergunta tem como seu enunciado ``Se encontrares um enfermeiro numa divisão, qual é a probabilidade de estar lá um doente?''.


\section{Conclusões}
\label{sec::implement:conc}

No presente capítulo foram apresentados os passos e os métodos necessários ao desenvolvimento do projeto prático e do funcionamento do mesmo. 

Desta forma, através do conteúdo exposto neste capítulo, encontra-se a apresentação do projeto desenvolvido e o funcionamento do mesmo, mas também a contextualização das ferramentas utilizadas conforme listadas no Capítulo ~\ref{ch::tecno}.