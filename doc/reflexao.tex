\chapter{Reflexão Crítica e Problemas Encontrados}
\label{ch::reflexao}

\section{Introdução}
\label{sec::reflexao:intro}

Neste Capítulo são explorados os seguintes tópicos:

\begin{itemize}
	\item Objetivos propostos vs. alcançados (Secção \ref{sec::reflexao:objetivos}): compara os objetivos inicialmente propostos com aqueles que foram concluídos no projeto prático;
	
	\item Divisão de trabalho pelos elementos do grupo (Secção \ref{sec::reflexao:divisao}): lista as tarefas divididas por cada elemento constituinte.
	
	\item Problemas encontrados (Secção \ref{sec::reflexao:problemas}): na sequência da Secção \ref{sec::reflexao:objetivos}, explora os problemas encontrados durante a implementação dos algoritmos;
	
	\item Reflexão crítica (Secção \ref{sec::reflexao:critica}): é feita uma \ac{SWOT} em retrospetiva pelo grupo acerca do projeto.
\end{itemize}


\section{Objetivos Propostos vs. Alcançados}
\label{sec::reflexao:objetivos}

A Tabela \ref{tab::objetivos} expõe os objetivos propostos inicialmente para o projeto e identifica quais foram alcançados totalmente ou parcialmente, e quais não foram bem sucedidos.

\begin{table}[!htbp]
	\centering
	\begin{tabular}{p{.35\textwidth} >{\centering\let\newline\\\arraybackslash\hspace{0pt}}m{.25\textwidth}}
		\toprule
		{\bfseries Objetivo proposto} & {\bfseries Alcançado?} \\
		\midrule
		Pergunta 1 & $\bullet$             \\
		Pergunta 2 & $\bullet$             \\
		Pergunta 3 & $\bullet$             \\
		Pergunta 4 & $\bullet$             \\
		Pergunta 5 & $\bullet$             \\
		Pergunta 6 & $\bullet$             \\
		Pergunta 7 & $\bullet$             \\
		Pergunta 8 & $\bullet$             \\
		Documentação do código & $\bullet$ \\
		\bottomrule
	\end{tabular}
	\caption[Objetivos propostos vs. alcançados]{
		Objetivos propostos e respetiva indicação de sucesso.\\
		\textit{Legenda.} $\bullet$ Totalmente alcançado; $\circ$ Parcialmente alcançado. -- Não alcançado.
	}
	\label{tab::objetivos}
\end{table}


\section{Divisão de Tarefas}
\label{sec::reflexao:divisao}

Para a gestão e divisão das tarefas que delineiam o projeto, foi realizada uma reunião inicial onde o foco incidiu na definição das metas por cada membro do grupo de forma balanceada. A referida gestão é apresentada na tabela ~\ref{tab::divisao-trabalho}. A escolha baseou-se no equilíbrio da dificuldade e temas das questões.

Não obstante, apesar da divisão de tarefas, houve uma essencial cooperação inicial entre os membros para definição das estruturas de dados e classes que se consideraram necessárias para a consolidação do projeto. Naturalmente houve de igual forma colaboração pontual no esclarecimento de dúvidas entre os membros.

A concretização do projeto foi conseguida com a realização de reuniões semanais, com \textit{deadlines} bem definidas para cada tarefa, tendo igualmente sido realizadas algumas reuniões extraordinárias em momentos críticos do desenvolvimento.

Por fim, o relatório e a apresentação foram divididos pelas diferentes partes que cada membro implementou, assim como pelos restantes Capítulos adicionais. A revisão final do relatório foi, contudo, conjunta a fim de garantir a sua coerência.

\begin{table}[!htbp]
	\centering
	\begin{tabular}{l c c}
		\toprule
		\textbf{Tarefas} & \textbf{Beatriz Costa} & \textbf{Igor Nunes}\\
		\midrule
		Pergunta 1      &   $\bullet$   &               \\
        Pergunta 2      &               &   $\bullet$   \\
        Pergunta 3      &   $\bullet$   &               \\
        Pergunta 4      &               &   $\bullet$   \\
        Pergunta 5      &   $\bullet$   &               \\
        Pergunta 6      &               &   $\bullet$   \\
        Pergunta 7      &   $\bullet$   &               \\
        Pergunta 8      &               &   $\bullet$   \\
        Relatório       &   $\bullet$   &   $\bullet$   \\
        Apresentação    &   $\bullet$   &   $\bullet$   \\
		\bottomrule
	\end{tabular}
	\caption[Distribuição de tarefas]{Distribuição de tarefas pelos elementos do grupo.}
	\label{tab::divisao-trabalho}
\end{table}


\section{Problemas Encontrados}
\label{sec::reflexao:problemas}

A implementação dos diferentes algoritmos necessários para responder às diversas perguntas colocadas levou a que fossem encontrados alguns problemas, os quais tiveram de ser ultrapassados a fim de terminar o projeto prático. Os problemas mais notáveis são resumidos na Tabela \ref{tab::problemas}, incluindo as soluções encontradas para os ultrapassar.

\begin{table}[!htbp]
	\centering
	\begin{tabular}{p{.46\textwidth} p{.46\textwidth}}
		\toprule
		{\bfseries Problema} & {\bfseries Solução} \\
		\midrule
		\midrule
		Ineficiência do algoritmo A* com recurso a uma matriz $800 \times 600$ & Implementação de dois grafos no lugar da matriz \\
		\midrule
		Determinação da relação entre a bateria, velocidade e tempo & Utilização de funções lineares para as respetivas estimativas \\
		\midrule
		Localização imprecisa dos objetos & Determinar a direção do \emph{robot} de forma a estimar a localização real do objeto \\
		\bottomrule
	\end{tabular}
	\caption[Problemas encontrados e respetivas soluções]{Problemas encontrados durante o desenvolvimento do projeto e respetivas soluções.}
	\label{tab::problemas}
\end{table}

De notar que, já na fase final do projeto, percebemos que existe a possibilidade de as relações entre a bateria e a velocidade com o tempo serem funções exponenciais. A sua implementação não seria complexa uma vez que a derivada e a primitiva de uma função exponencial do tipo $ke^x$ é a própria função exponencial (propriedade importante para a pergunta 5 (secção \ref{ssec::implement:details:perg5})):



\section{Reflexão Crítica}
\label{sec::reflexao:critica}

É proposto expor a reflexão crítica face ao trabalho realizado para o desenvolvimento deste projeto através de uma análise \ac{SWOT}.


\subsection{Pontos Fortes}
\label{ssec::reflexao:critica:fortes}

\begin{enumerate}
	\item Forte estruturação dos tipos de dados com recurso a classes;
	
	\item Utilização dos grafos para aumento significativo da eficiência;
	
	\item Implementação de um método altamente eficiente para estimar a duração da bateria e prever a velocidade.
\end{enumerate}


\subsection{Pontos Fracos}
\label{ssec::reflexao:critica:fracos}

\begin{enumerate}
	\item O código-fonte final, incluindo documentação, é denso e de difícil navegação;
    
    \item As funções lineares que relacionam a bateria e a velocidade com o tempo não representam fielmente a variação que ocorre ao longo do tempo;
    
    \item Várias funções não foram otimizadas a fim de melhorar o consumo de recursos e/ou de aumentar a eficiência temporal.
\end{enumerate}


\subsection{Ameaças}
\label{ssec::reflexao:critica:ameacas}

\begin{enumerate}
	\item Em efeitos de expansão do número de \emph{robots} a circular, as classes com métodos estáticos não permitem a criação de instâncias independentes para cada \emph{robot}, ficando assim um piso limitado apenas a um \emph{robot};
	
	\item O uso de grafos não permite obter o caminho exato que o \emph{robot} deve fazer pelo mundo até determinada sala, sendo por conseguinte imprecisa a distância calculada para lá chegar;
	
	\item O método de estimação da bateria do \emph{robot} é impreciso uma vez que o comportamento real destas funções não se revela linear.
\end{enumerate}


\subsection{Oportunidades}
\label{ssec::reflexao:critica:oportunidades}

\begin{enumerate}
	\item Reformular a organização do código de forma a permitir a divisão de ficheiros por classes;
    
    \item Estudar melhor a variação da bateria e da velocidade em relação ao tempo a fim de auferir se se tratam de funções exponenciais;
    
    \item Estudar os melhores métodos de otimização de cada função a fim de evitar processos potencialmente redundantes.
\end{enumerate}


\section{Conclusões}
\label{sec::reflexao:conclusao}

Esta fase de reflexão permitiu analisar o trabalho levado ao longo das semanas de planeamento e implementação dos algoritmos. Com esta análise, o grupo pôde tirar conclusões acerca das estratégias utilizadas, as quais serão expostas no Capítulo seguinte.
